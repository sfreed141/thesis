\chapter{Related Work}

Global illumination is a broad term that can be approached in many ways. With respect to this thesis, we only focus on comparisons with other real-time, fully dynamic techniques are used.

* Global Illumination using Voxel Cone Tracing
    - octrees vs clipmaps vs 3d textures (vs warped)
    - proprietary
    - vxgi: clipmaps, in UE4 but still closed source
    - octree -> no hw filtering, complex but higher resolution
    - multiple levels of radiance -> allows gathering more high frequency indirect lighting information

* Light Propogation Volumes
    - in cryengine (closed source (check that); difficult to understand within context of engine)
    - somewhat like vct but only one level of detail is stored (filtered completely)
        - low frequency only (no specular)
        - can't be used for other purposes like occlusion
    - no voxelization (no geometric information)
    - no cone tracing needed (just samples texture directly)

* from GPU Pro 4
    - VPL injection using instanced quads (instead of RSMs) -> not as accurate but faster

* some common issues with other works (this could also go in intro)
    - closed source
        - implementation details matter -> difficult to understand from research or white papers
        - highly variable performance based on unrelated things (OpenGL vs DirectX? modern OpenGL? mipmaps, normal maps, shading model, mesh optimizations, etc)
    - research renderers or full game engines -> difficult to understand